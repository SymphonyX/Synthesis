\documentclass[a4paper]{article}
\usepackage[latin1]{inputenc}
\usepackage{graphicx}
\usepackage{array}
\usepackage{amsmath}
\newcommand{\transp}{\top}

\setlength{\parskip}{2ex}

\begin{document}


\title{Independent Study Report - CB + DMPs}
\author{Francisco M. Garcia}
%\date{}

\maketitle


\section{Introduction}

So far we 

\section{Mixture of Experts - Softmax}

In this se

\section{Dynamic Movement Primitives}

Dynamic Movement Primitives (DMPs) provide a general approach for learning robotic motor skill from demonstration. Given a sample trajectory with start state $x_0$ and goal state $g$, a DMP generates a trajectory by integrating the following equations:

$$
\tau \dot{v} = K(g - x) - Dv - K(g - x_0) s + Kf(s)
$$
$$
\tau \dot{x} = v
$$

In these equations, $\tau$ refers to a time scaling term, $K$ and $D$ refer on system specific constant that work as in a PD controller and $f(s)$ corresponds to a non-linear function composed of several Gaussian basis functions:

$$
f(s) = \frac{\sum_i w_i \psi_i(s) s }{\sum_i \psi_i(s)}
$$

where $\psi_i(s) = \exp(-h_i(s-c_i)^2)$ are the basis functions with center $c_i$ and width $h_i$. $w_i$ are the parameters to be found to minimize the objective function J. \\
\indent The variable $s$ is a phase variable which encompases the duration of the trajectory and monotonically decreases from 1 to 0. This variable obtained by the canonical system:
$$
\tau \dot{s} = -\alpha s
$$
 
When observing a demonstration, a movement $x(t)$ is recorded and from that its derivative $v(t)$ and $\dot{v}$ are obtained. Using this, we can compute the target function:
$$
f_{target}(s) = \frac{\tau \dot{v} + Dv}{K} - (g - x) + (g - x_0) s
$$ 

and solve the objective function $J = \sum_s (f_{target}(s) - f(s))^2$ via gradient descent or least-squares. \\
\indent Once trained, a new trajectory can be generated by setting $s$ to 1, updating the positions and velocities and intergrating the canonical system to obtained the new $s$.
 
 
 
\end{document}